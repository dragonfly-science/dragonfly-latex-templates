% This is an example of using the dragonfly-letter class. Note that when writing
% a letter you should consider whether the letter template is actually appropriate
% for your task. As for all templates letter accepts a draft option which will 
% insert a watermark, increase the linespacing and add in line numbers
\documentclass{dragonfly-letter}

% The lipsum package is used to generate filler text and should not be included
% in a real letter
\usepackage{lipsum}
 
% Every letter has a compulsory title
\title{An Non optional Title}

% Every letter has a compulsory Author
\author{An Non optional Example Author}

% Each letter starts with a greeting. At the time of writing the standard is to 
% not place any punctuation at the end of the greeting line.
\greeting{Dear Anthony}

% The preamble goes at the top of the letter. The first argument is the date of the 
% letter, while the second is the address of the recipient. Note that the
% assess should be written out without using any latex macros to control the spacing.
% The linebreaks used will be preserved exactly within the preamble command.
\preamble{\today}{
 67 Fox Road
 Narnia
 California 9876
 China
 }

% At this point the letter begins. This will also create the letter head.
\begin{document}

% The letter content goes here. This command generates some filler text. 
\lipsum[1-7]

% The signature is used to place an ending that has a space for a person to sign
% it. Additionally it can be given an image to use as a signature.
% In which case it is used as \signature{Regards}[imagefile]{Ending}
% Note that line breaks are preserved in the signature for ease of writing
% multiline endings.
\signature{Cheers}{
A Person
But an important one
With a multi line
Signature
}

% We place text after the signature if we require. 
P.S. Look a postscript in postscript!

%Nothing beyond this point will be processed as part of the document. 
\end{document}
